General Purpose

The term cluster analysis (first used by Tryon, 1939) encompasses a number of different algorithms and methods for grouping objects of similar kind into respective categories. A general question facing researchers in many areas of inquiry is how to organize observed data into meaningful structures, that is, to develop taxonomies. In other words cluster analysis is an exploratory data analysis tool which aims at sorting different objects into groups in a way that the degree of association between two objects is maximal if they belong to the same group and minimal otherwise. Given the above, cluster analysis can be used to discover structures in data without providing an explanation/interpretation. In other words, cluster analysis simply discovers structures in data without explaining why they exist.

We deal with clustering in almost every aspect of daily life. For example, a group of diners sharing the same table in a restaurant may be regarded as a cluster of people. In food stores items of similar nature, such as different types of meat or vegetables are displayed in the same or nearby locations. There is a countless number of examples in which clustering plays an important role. For instance, biologists have to organize the different species of animals before a meaningful description of the differences between animals is possible. According to the modern system employed in biology, man belongs to the primates, the mammals, the amniotes, the vertebrates, and the animals. Note how in this classification, the higher the level of aggregation the less similar are the members in the respective class. Man has more in common with all other primates (e.g., apes) than it does with the more "distant" members of the mammals (e.g., dogs), etc. For a review of the general categories of cluster analysis methods, see Joining (Tree Clustering), Two-way Joining (Block Clustering), and k-Means Clustering. In short, whatever the nature of your business is, sooner or later you will run into a clustering problem of one form or another.

\subsection*{Statistical Significance Testing}

Note that the above discussions refer to clustering algorithms and do not mention anything about statistical significance testing. In fact, cluster analysis is not as much a typical statistical test as it is a "collection" of different algorithms that "put objects into clusters according to well defined similarity rules." The point here is that, unlike many other statistical procedures, cluster analysis methods are mostly used when we do not have any a priori hypotheses, but are still in the exploratory phase of our research. In a sense, cluster analysis finds the "most significant solution possible." Therefore, statistical significance testing is really not appropriate here, even in cases when p-levels are reported (as in k-means clustering).

Area of Application

Clustering techniques have been applied to a wide variety of research problems. Hartigan (1975) provides an excellent summary of the many published studies reporting the results of cluster analyses. For example, in the field of medicine, clustering diseases, cures for diseases, or symptoms of diseases can lead to very useful taxonomies. In the field of psychiatry, the correct diagnosis of clusters of symptoms such as paranoia, schizophrenia, etc. is essential for successful therapy. In archeology, researchers have attempted to establish taxonomies of stone tools, funeral objects, etc. by applying cluster analytic techniques. In general, whenever we need to classify a "mountain" of information into manageable meaningful piles, cluster analysis is of great utility.

To index

 
Joining (Tree Clustering)

Hierarchical Tree
Distance Measures
Amalgamation or Linkage Rules
GENERAL LOGIC

The example in the General Purpose Introduction illustrates the goal of the joining or tree clustering algorithm. The purpose of this algorithm is to join together objects (e.g., animals) into successively larger clusters, using some measure of similarity or distance. A typical result of this type of clustering is the hierarchical tree.

HIERARCHICAL TREE

Consider a Horizontal Hierarchical Tree Plot (see graph below), on the left of the plot, we begin with each object in a class by itself. Now imagine that, in very small steps, we "relax" our criterion as to what is and is not unique. Put another way, we lower our threshold regarding the decision when to declare two or more objects to be members of the same cluster.



As a result we link more and more objects together and aggregate (amalgamate) larger and larger clusters of increasingly dissimilar elements. Finally, in the last step, all objects are joined together. In these plots, the horizontal axis denotes the linkage distance (in Vertical Icicle Plots, the vertical axis denotes the linkage distance). Thus, for each node in the graph (where a new cluster is formed) we can read off the criterion distance at which the respective elements were linked together into a new single cluster. When the data contain a clear "structure" in terms of clusters of objects that are similar to each other, then this structure will often be reflected in the hierarchical tree as distinct branches. As the result of a successful analysis with the joining method, we are able to detect clusters (branches) and interpret those branches.

\subsection*{DISTANCE MEASURES}

The joining or tree clustering method uses the dissimilarities (similarities) or distances between objects when forming the clusters. Similarities are a set of rules that serve as criteria for grouping or separating items. In the previous example the rule for grouping a number of dinners was whether they shared the same table or not. These distances (similarities) can be based on a single dimension or multiple dimensions, with each dimension representing a rule or condition for grouping objects. For example, if we were to cluster fast foods, we could take into account the number of calories they contain, their price, subjective ratings of taste, etc. The most straightforward way of computing distances between objects in a multi-dimensional space is to compute Euclidean distances. If we had a two- or three-dimensional space this measure is the actual geometric distance between objects in the space (i.e., as if measured with a ruler). However, the joining algorithm does not "care" whether the distances that are "fed" to it are actual real distances, or some other derived measure of distance that is more meaningful to the researcher; and it is up to the researcher to select the right method for his/her specific application.

Euclidean distance. This is probably the most commonly chosen type of distance. It simply is the geometric distance in the multidimensional space. It is computed as:

distance(x,y) = {i (xi - yi)2 }½

Note that Euclidean (and squared Euclidean) distances are usually computed from raw data, and not from standardized data. This method has certain advantages (e.g., the distance between any two objects is not affected by the addition of new objects to the analysis, which may be outliers). However, the distances can be greatly affected by differences in scale among the dimensions from which the distances are computed. For example, if one of the dimensions denotes a measured length in centimeters, and you then convert it to millimeters (by multiplying the values by 10), the resulting Euclidean or squared Euclidean distances (computed from multiple dimensions) can be greatly affected (i.e., biased by those dimensions which have a larger scale), and consequently, the results of cluster analyses may be very different. Generally, it is good practice to transform the dimensions so they have similar scales.

Squared Euclidean distance. You may want to square the standard Euclidean distance in order to place progressively greater weight on objects that are further apart. This distance is computed as (see also the note in the previous paragraph):

distance(x,y) = i (xi - yi)2

City-block (Manhattan) distance. This distance is simply the average difference across dimensions. In most cases, this distance measure yields results similar to the simple Euclidean distance. However, note that in this measure, the effect of single large differences (outliers) is dampened (since they are not squared). The city-block distance is computed as:

distance(x,y) = i |xi - yi|

Chebychev distance. This distance measure may be appropriate in cases when we want to define two objects as "different" if they are different on any one of the dimensions. The Chebychev distance is computed as:

distance(x,y) = Maximum|xi - yi|

Power distance. Sometimes we may want to increase or decrease the progressive weight that is placed on dimensions on which the respective objects are very different. This can be accomplished via the power distance. The power distance is computed as:

distance(x,y) = (i |xi - yi|p)1/r

where r and p are user-defined parameters. A few example calculations may demonstrate how this measure "behaves." Parameter p controls the progressive weight that is placed on differences on individual dimensions, parameter r controls the progressive weight that is placed on larger differences between objects. If r and p are equal to 2, then this distance is equal to the Euclidean distance.

Percent disagreement. This measure is particularly useful if the data for the dimensions included in the analysis are categorical in nature. This distance is computed as:

distance(x,y) = (Number of xi  yi)/ i

\subsection*{AMALGAMATION OR LINKAGE RULES}

At the first step, when each object represents its own cluster, the distances between those objects are defined by the chosen distance measure. However, once several objects have been linked together, how do we determine the distances between those new clusters? In other words, we need a linkage or amalgamation rule to determine when two clusters are sufficiently similar to be linked together. There are various possibilities: for example, we could link two clusters together when any two objects in the two clusters are closer together than the respective linkage distance. Put another way, we use the "nearest neighbors" across clusters to determine the distances between clusters; this method is called single linkage. This rule produces "stringy" types of clusters, that is, clusters "chained together" by only single objects that happen to be close together. Alternatively, we may use the neighbors across clusters that are furthest away from each other; this method is called complete linkage. There are numerous other linkage rules such as these that have been proposed.

Single linkage (nearest neighbor). As described above, in this method the distance between two clusters is determined by the distance of the two closest objects (nearest neighbors) in the different clusters. This rule will, in a sense, string objects together to form clusters, and the resulting clusters tend to represent long "chains."

Complete linkage (furthest neighbor). In this method, the distances between clusters are determined by the greatest distance between any two objects in the different clusters (i.e., by the "furthest neighbors"). This method usually performs quite well in cases when the objects actually form naturally distinct "clumps." If the clusters tend to be somehow elongated or of a "chain" type nature, then this method is inappropriate.

Unweighted pair-group average. In this method, the distance between two clusters is calculated as the average distance between all pairs of objects in the two different clusters. This method is also very efficient when the objects form natural distinct "clumps," however, it performs equally well with elongated, "chain" type clusters. Note that in their book, Sneath and Sokal (1973) introduced the abbreviation UPGMA to refer to this method as unweighted pair-group method using arithmetic averages.

Weighted pair-group average. This method is identical to the unweighted pair-group average method, except that in the computations, the size of the respective clusters (i.e., the number of objects contained in them) is used as a weight. Thus, this method (rather than the previous method) should be used when the cluster sizes are suspected to be greatly uneven. Note that in their book, Sneath and Sokal (1973) introduced the abbreviation WPGMA to refer to this method as weighted pair-group method using arithmetic averages.

Unweighted pair-group centroid. The centroid of a cluster is the average point in the multidimensional space defined by the dimensions. In a sense, it is the center of gravity for the respective cluster. In this method, the distance between two clusters is determined as the difference between centroids. Sneath and Sokal (1973) use the abbreviation UPGMC to refer to this method as unweighted pair-group method using the centroid average.

Weighted pair-group centroid (median). This method is identical to the previous one, except that weighting is introduced into the computations to take into consideration differences in cluster sizes (i.e., the number of objects contained in them). Thus, when there are (or we suspect there to be) considerable differences in cluster sizes, this method is preferable to the previous one. Sneath and Sokal (1973) use the abbreviation WPGMC to refer to this method as weighted pair-group method using the centroid average.

Ward's method. This method is distinct from all other methods because it uses an analysis of variance approach to evaluate the distances between clusters. In short, this method attempts to minimize the Sum of Squares (SS) of any two (hypothetical) clusters that can be formed at each step. Refer to Ward (1963) for details concerning this method. In general, this method is regarded as very efficient, however, it tends to create clusters of small size.

For an overview of the other two methods of clustering, see Two-way Joining and k-Means Clustering.

To index

 
Two-Way Joining

%Introductory Overview
%Two-way Joining
%INTRODUCTORY OVERVIEW

Previously, we have discussed this method in terms of "objects" that are to be clustered (see Joining (Tree Clustering)). In all other types of analyses the research question of interest is usually expressed in terms of cases (observations) or variables. It turns out that the clustering of both may yield useful results. For example, imagine a study where a medical researcher has gathered data on different measures of physical fitness (variables) for a sample of heart patients (cases). The researcher may want to cluster cases (patients) to detect clusters of patients with similar syndromes. At the same time, the researcher may want to cluster variables (fitness measures) to detect clusters of measures that appear to tap similar physical abilities.

\subsection{TWO-WAY JOINING}

Given the discussion in the paragraph above concerning whether to cluster cases or variables, we may wonder why not cluster both simultaneously? Two-way joining is useful in (the relatively rare) circumstances when we expect that both cases and variables will simultaneously contribute to the uncovering of meaningful patterns of clusters.



For example, returning to the example above, the medical researcher may want to identify clusters of patients that are similar with regard to particular clusters of similar measures of physical fitness. The difficulty with interpreting these results may arise from the fact that the similarities between different clusters may pertain to (or be caused by) somewhat different subsets of variables. Thus, the resulting structure (clusters) is by nature not homogeneous. This may seem a bit confusing at first, and, indeed, compared to the other clustering methods described (see Joining (Tree Clustering) and k-Means Clustering), two-way joining is probably the one least commonly used. However, some researchers believe that this method offers a powerful exploratory data analysis tool (for more information you may want to refer to the detailed description of this method in Hartigan, 1975).

%================================================================================================================= %