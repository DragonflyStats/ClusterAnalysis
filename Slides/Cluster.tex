\documentclass[]{article}

%opening
\title{}
\author{}

\usepackage{framed}


\begin{document}

\maketitle

\begin{abstract}

\end{abstract}

%----------------------------------------%

\subsection{Clustering Analysis with \texttt{R}}
\Large
Data set for this example
\begin{framed}
\begin{verbatim}
X <-iris[ , 1:4 ]

\end{verbatim}
\end{framed}




\Large
\begin{itemize}
\item \texttt{dist()} - computes the \textit{\textbf{distance matrix}} i.e. a matrix of similarity measures for each pair of cases in the data set.
\vspace{0.3cm}
\item \texttt{hclust()} - constructs a hierarchical clustering procedure, based on one of the linkage procedures.
\vspace{0.3cm}
\end{itemize}


\begin{itemize}

\item \texttt{rect.hclust()}: This command draws rectangles around the branches of a dendrogram highlighting the corresponding clusters. \item First the dendrogram is cut at a certain height (\texttt{h}) level, then a rectangle is drawn around selected branches.
\end{itemize}


\begin{framed}
\begin{verbatim}
plot(hclust(DistMat,method="ward"))
rect.hclust(hclust(DistMat,method="ward"),h=10)

plot(hclust(DistMat,method="centroid"))
rect.hclust(hclust(DistMat,method="centroid"),h=1.2)

plot(hclust(DistMat,method="average"))
rect.hclust(hclust(DistMat,method="average"),h=1.2)
\end{verbatim}
\end{framed}



%---------------------------------------------------------- %


\subsection{k-means clustering with \texttt{R}}

For this example, we will use the four numeric columns from the iris data set, i.e. columns 1 to 4.
\begin{framed}
\begin{verbatim}
X <- iris[,1:4]
\end{verbatim}
\end{framed}


%---------------------------------------------------------- %


\subsection{k-means clustering with \texttt{R}}

\begin{framed}
\begin{verbatim}
kmeans(X,centers=3)
\end{verbatim}
\end{framed}


%---------------------------------------------------------- %
%---------------------------------------------------------- %


\subsection{k-means clustering with \texttt{R}}

\begin{framed}
\begin{verbatim}
kmeans(X,centers=3)
\end{verbatim}
\end{framed}
\begin{verbatim}
Within cluster sum of squares by cluster:
[1] 15.15100 39.82097 23.87947
 (between_SS / total_SS =  88.4 %)
\end{verbatim}

%---------------------------------------------------------- %

\Large
\begin{framed}
\begin{verbatim}
PredClass <- kmeans(X,centers=3)$cluster
table(PredClass,iris$Species)
\end{verbatim}
\end{framed}
\begin{verbatim}
        
PredClass setosa versicolor virginica
        1     50          0         0
        2      0          2        36
        3      0         48        14

\end{verbatim}
\end{document}
