
\documentclass[a4paper,12pt]{article}
%%%%%%%%%%%%%%%%%%%%%%%%%%%%%%%%%%%%%%%%%%%%%%%%%%%%%%%%%%%%%%%%%%%%%%%%%%%%%%%%%%%%%%%%%%%%%%%%%%%%%%%%%%%%%%%%%%%%%%%%%%%%%%%%%%%%%%%%%%%%%%%%%%%%%%%%%%%%%%%%%%%%%%%%%%%%%%%%%%%%%%%%%%%%%%%%%%%%%%%%%%%%%%%%%%%%%%%%%%%%%%%%%%%%%%%%%%%%%%%%%%%%%%%%%%%%
\usepackage{eurosym}
\usepackage{vmargin}
\usepackage{amsmath}
\usepackage{graphics}
\usepackage{epsfig}
\usepackage{subfigure}
\usepackage{fancyhdr}
%\usepackage{listings}
\usepackage{framed}
\usepackage{graphicx}

\setcounter{MaxMatrixCols}{10}
%TCIDATA{OutputFilter=LATEX.DLL}
%TCIDATA{Version=5.00.0.2570}
%TCIDATA{<META NAME="SaveForMode" CONTENT="1">}
%TCIDATA{LastRevised=Wednesday, February 23, 2011 13:24:34}
%TCIDATA{<META NAME="GraphicsSave" CONTENT="32">}
%TCIDATA{Language=American English}

\pagestyle{fancy}
\setmarginsrb{20mm}{0mm}{20mm}{25mm}{12mm}{11mm}{0mm}{11mm}
\lhead{MA4128} \rhead{Mr. Kevin O'Brien}
\chead{Advanced Data Modelling}
%\input{tcilatex}


% http://www.norusis.com/pdf/SPC_v13.pdf
\begin{document}
\section{Standardizing the Variables}
If variables are measured on different scales, variables with large values contribute
more to the distance measure than variables with small values. In this example, both
variables are measured on the same scale, so that’s not much of a problem, assuming
the judges use the scales similarly. But if you were looking at the distance between two
people based on their IQs and incomes in dollars, you would probably find that the
differences in incomes would dominate any distance measures. (A difference of only
\$100 when squared becomes 10,000, while a difference of 30 IQ points would be only
900. I’d go for the IQ points over the dollars!).

Variables that are measured in large numbers will contribute to the distance more than variables recorded in smaller
numbers.

In the hierarchical clustering procedure in SPSS, you can standardize variables in
different ways. You can compute standardized scores or divide by just the standard
deviation, range, mean, or maximum. This results in all variables contributing more
equally to the distance measurement. That’s not necessarily always the best strategy,
since variability of a measure can provide useful information. 
%----------------------------------------------------- %
\end{document}