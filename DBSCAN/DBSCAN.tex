Density-based spatial clustering of applications with noise (DBSCAN) is a data clustering algorithm proposed by Martin Ester, Hans-Peter Kriegel, Jörg Sander and Xiaowei Xu in 1996.[1] It is a density-based clustering algorithm: given a set of points in some space, it groups together points that are closely packed together (points with many nearby neighbors), marking as outliers points that lie alone in low-density regions (whose nearest neighbors are too far away). 
DBSCAN is one of the most common clustering algorithms and also most cited in scientific literature.[2]


%-----------------------------------------------------------------------------------%
Advantages
DBSCAN does not require one to specify the number of clusters in the data a priori, as opposed to k-means.
DBSCAN can find arbitrarily shaped clusters. It can even find a cluster completely surrounded by (but not connected to) a different cluster. Due to the MinPts parameter, the so-called single-link effect (different clusters being connected by a thin line of points) is reduced.
DBSCAN has a notion of noise, and is robust to outliers.
DBSCAN requires just two parameters and is mostly insensitive to the ordering of the points in the database. (However, points sitting on the edge of two different clusters might swap cluster membership if the ordering of the points is changed, and the cluster assignment is unique only up to isomorphism.)
DBSCAN is designed for use with databases that can accelerate region queries, e.g. using an R* tree.
The parameters minPts and ε can be set by a domain expert, if the data is well understood.
%-----------------------------------------------------------------------------------%

Disadvantages
DBSCAN is not entirely deterministic: border points that are reachable from more than one cluster can be part of either cluster, depending on the order the data is processed. Fortunately, this situation does not arise often, and has little impact on the clustering result: both on core points and noise points, DBSCAN is deterministic. DBSCAN*[4] is a variation that treats border points as noise, and this way achieves a fully deterministic result as well as a more consistent statistical interpretation of density-connected components.
The quality of DBSCAN depends on the distance measure used in the function regionQuery(P,ε). The most common distance metric used is Euclidean distance. Especially for high-dimensional data, this metric can be rendered almost useless due to the so-called "Curse of dimensionality", making it difficult to find an appropriate value for ε. This effect, however, is also present in any other algorithm based on Euclidean distance.
DBSCAN cannot cluster data sets well with large differences in densities, since the minPts-ε combination cannot then be chosen appropriately for all clusters.
If the data and scale are not well understood, choosing a meaningful distance threshold ε can be difficult.
See the section below on extensions for algorithmic modifications to handle these issues.
